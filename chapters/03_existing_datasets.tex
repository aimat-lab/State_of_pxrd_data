To contextualize opXRD within the current environment of experimental powder diffraction data, the below list provides an overview of the largest crystal structure databases that offer access to experimental powder diffraction data. For an overview of these databases refer to Table$\eqref{tab:exp_databases}$ below. In this table, the column ``O.A.'' indicates whether or not the database is open access. The availability of the chemical composition, spacegroups, lattice parameters, and atomic coordinates of the underlying samples are indicated by the columns ``Comp.'', ``Spg.'', ``Lattice'' and ``Atom coords.'' respectively.

\begin{table}[!htb]
\centering
\caption{Overview of experimental powder diffraction databases}
\label{tab:exp_databases}
\scalebox{0.84}{
\begin{adjustbox}{center}
\begin{NiceTabular}{@{}llllllll@{}}
\toprule
\textbf{Name} & \textbf{No. patterns} & \textbf{O.A.} & \textbf{Comp.} & \textbf{Spg.} & \textbf{Lattice} & \textbf{Atom coords.} & \textbf{Year est.} \\
\midrule
Linus Pauling file                     & 21,700                    & \xmark          & \cmark        & \cmark        & \cmark        & \cmark          & 2002            \\
Powder Diffraction File \tabularnote[a)]{The PDF lists the Material Platform for Data Science (MPDS) as a database source. Since the MPDS is hosted by the Pauling File project, there is likely significant overlap in the experimental patterns available in the PDF and the Linus Pauling File.} & 20,800 & \xmark & \cmark & \cmark & \cmark & \partialcheck{52} & 1941 \\
RRUFF                                  & 1290                      & \cmark          & \cmark        & \cmark        & \cmark        & \xmark          & 2006            \\
Crystallography Open Database          & 1052                      & \cmark          & \cmark        & \cmark        & \partialcheck{85}        & \partialcheck{85}          & 2003            \\
PowBase                                & 169                       & \cmark          & \cmark        & \xmark        & \xmark        & \xmark          & 1999            \\
\bottomrule
\end{NiceTabular}
\end{adjustbox}}
\end{table}

\textbf{Linus Pauling File}:\cite{PaulingWeb} The Linus Pauling File is a largely commercial crystal structure database published and maintained by the Pauling File project \cite{villars2018}. It is currently distributed as Pearson Crystal data \cite{PearsonWeb} and the Materials Platform for Data Science (MPDS)\cite{MPDSWeb}. The database, first published in 2002, currently contains more than 534,000 crystal structures\cite{MPDSWeb} and 21,700 corresponding experimental powder diffraction patterns\cite{PearsonWeb}. 
This makes the Pauling file, to the best of our knowledge, the largest collection of experimental powder diffraction data available to researchers. As of November 2024, Pearson's crystal data is available to researchers through a purchase of a one-year license starting at a price point of \qty{2200}{\myeuro}\cite{PearsonBuy}. The MPDS is partially open, with the open MPDS data accessible through a web interface\cite{MPDSWeb}. API access to the full MPDS can be purchased through a one-year license starting at \qty{9500}{\myeuro}\cite{MPDSBuy}. When inquired about the availability of experimental powder diffraction data in the MPDS API, the Pauling File project stated that this data is not currently included. It could however be offered in the future at the request of customers. \\

\textbf{Powder Diffraction File:} \cite{PDFWeb} The Powder Diffraction File (PDF), published and maintained by the International Center for Diffraction Data (ICDD), is a large collection of materials with accompanying powder diffraction data first published in 1941\cite{GatesRector2019}. According to the ICDD the PDF5+, the latest release of the PDF as of November 2024, contains over a million materials with accompanying powder diffraction data. However, most of these powder diffraction patterns are simulated. After inquiring with the ICDD in April 2024 we were told that only 20,800 of the powder diffraction patterns in the PDF5+ stemmed from experiments. Of these 20,800 entries, 10,954 contain information about the atomic coordinates of the underlying structures. Since the PDF5+ lists the MPDS as a database source, there is likely a significant overlap in the experimental patterns found in the PDF5+ and those found in the Pauling file. As of November 2024, the PDF5+ is available to researchers through a purchase of a one-year license starting at a price point of \$6265.00. However, the ICDD does not allow researchers to train machine learning models on PDF5+ data, regardless of whether the resulting models are published.\footnote{This information can be found in the product license agreement for the PDF5+ at \url{https://www.icdd.com/licensing-process/\#1528471154226-933e5cc6-8da7}.} \\

\textbf{RRUFF}: \cite{RRUFFWeb} The RRUFF Mineral Database, first published in 2006, provides detailed information on minerals, including their chemical compositions, crystallography, and spectroscopic data. \cite{lafuente2015} Managed by the University of Arizona, it was created to serve as a public repository for mineral identification and research. It contains \num{1290} powder diffraction patterns stemming from experiments each labeled with the lattice parameters and composition of the underlying structures. The RRUFF data is openly accessible on the official website \cite{RRUFFWeb}. \\

\textbf{Crystallography Open Database:} \cite{CODWeb} The Crystallography Open Database (COD) is an open-access collection of crystal structures founded in 2003\cite{Graulis2009cod}. It currently provides over 500,000 crystal structures. Of these files, 1052 contains the experimental powder diffraction data that was used to determine the underlying crystal structures of the investigated samples. Hence, the experimental powder diffraction data contained in the COD is mostly labeled with the full crystal structure information. The data is openly accessible in the form of .cif files on the official COD website\cite{CODWeb}. \\

\textbf{PowBase}:\footnote{More information on PowBase can be found at \url{http://www.cristal.org/powbase/index.html}. PowBase was an initiative suggested in the Structure Determination by Powder Diffractometry (SDPD) mailing list which was hosted on the same site. The COD was another community initiative that grew out of this mailing list.} PowBase is a database of 169 mostly unlabeled experimental powder diffraction patterns collected and maintained by crystallography researcher Armel Le Bail starting in 1999. As of November 2024, all 169 patterns are still freely available for download. \\

There is also publicly available powder diffraction data uploaded to datasets on Zenodo. However, this data is split into disparate entries that typically only contain the work of a single research project. Additionally, extracting powder diffraction data at scale is hindered by the fact that the data is often given in plain text files in non-standardized formats, which are difficult to parse programmatically. We are currently planning a systematic large-scale extraction of powder diffraction data from databases like Zenodo with the help of a large language model. We will include this data in a future release of our dataset.\\

Aside from the databases mentioned above, we have also investigated several other crystal structure resources in search of experimental powder diffraction data. Crystal structure resources that were investigated but not found to contain any appreciable amount of publicly available experimental powder diffraction data include the Inorganic Crystal Structure Database \cite{ICSDWeb}, the Cambridge Structural Database \cite{CambridgeWeb}, the Materials Project database \cite{MatProjWeb}, the Crystallographic and Crystallochemical Database \cite{CrystallochemicalWeb}, the Bilbao Incommensurate Crystal Structure Database \cite{BilbaoWeb}, the Mineralogy Database \cite{MineralogyWeb}, the IUCr Raw data letters \cite{IUCrWeb}, the U.S. Naval Research Laboratory Crystal Lattice-Structures \cite{NRLWeb}, the Athena Mineral database \cite{AthenaWeb} and the Protein data bank\cite{PDBWeb}. This is to be expected as most structure solutions are achieved through single-crystal diffraction rather than powder diffraction. \\ 