\paragraph{ML for pXRD} Several approaches have applied machine learning methods to classification and regression tasks for powder diffractograms.

Using simulated diffractograms based on structures from the ICSD, Lee et al. managed to train a deep convolutional neural network (CNN) that is able to classify occurring phases in diffractograms of a specific compound pool. \cite{Lee2020}
Park et al. reached a test accuracy of roughly 81\% for a CNN, which classifies space groups of simulated single-phase diffractograms. \cite{Park2017}
A regression analysis on lattice parameters within a broader framework encompassing all material classes was conducted by Chitturi et al.\cite{Chitturi2021} They developed a distinct CNN for each crystal system, utilizing a merged dataset from both the ICSD and the Cambridge Structural Database, and managed to achieve a mean absolute percentage error of about \SI{10}{\percent} for the lattice lengths, although they encountered difficulties in accurately predicting angles.

TODO: We need a short literature review here; needs to also cover papers published after our previous paper!

TODO: briefly discuss here that there are expert models for specific classes of materials which give more information, and more generic models like ours which currently only gives you the space group and not much more; especially focus on how experimental conditions (background, noise, ... is modeled)
Phase Labeling and Lattice Refinement: https://arxiv.org/abs/2308.07897
crystal-structure phase mapping: https://www.nature.com/articles/s42256-021-00384-1


TODO: briefly discuss papers about multiphase classification and remixing of diffraction patterns
Non-negative Matrix Factorization 2018: https://link.springer.com/chapter/10.1007/978-3-319-93031-2_4
non-negative matrix factorization 2018: https://ojs.aaai.org/aimagazine/index.php/aimagazine/article/view/2785
Physically-informed Graph-based DRNet (PG-DRNet) 2023: https://ieeexplore.ieee.org/abstract/document/10459774/

TODO: discuss in more detail why domain randomization as we did in the last paper so far failed for background and noise, and thus why we need background and noise statistics from experiment to enhance simulated datasets and transfer them to the experimental data distribution.