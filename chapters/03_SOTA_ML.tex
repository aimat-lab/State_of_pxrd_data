\paragraph{ML for pXRD} Several approaches have applied machine learning methods to classification and regression tasks for powder diffractograms.

TODO: We need a short literature review here; needs to also cover papers published after our previous paper!

TODO: briefly discuss here that there are expert models for specific classes of materials which give more information, and more generic models like ours which currently only gives you the space group and not much more; especially focus on how experimental conditions (background, noise, ... is modeled)
Phase Labeling and Lattice Refinement: https://arxiv.org/abs/2308.07897


TODO: briefly discuss papers about multiphase classification and remixing of diffraction patterns
Physically-informed Graph-based DRNet (PG-DRNet): https://ieeexplore.ieee.org/abstract/document/10459774/
https://www.nature.com/articles/s42256-021-00384-1
non-negative matrix factorization: https://ojs.aaai.org/aimagazine/index.php/aimagazine/article/view/2785

TODO: discuss in more detail why domain randomization as we did in the last paper so far failed for background and noise, and thus why we need background and noise statistics from experiment to enhance simulated datasets and transfer them to the experimental data distribution.