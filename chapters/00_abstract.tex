Powder X-ray diffraction (pXRD) experiments are a cornerstone for materials structure characterization.
Despite their widespread application, the analysis of pXRD diffractograms still presents a significant challenge to automation and thus a bottleneck in high-throughput materials discovery in self-driving labs.
Machine learning has emerged as a promising research direction to resolve this bottleneck by enabling automated powder diffraction data analysis.
A notable difficulty in applying machine learning to this domain is the lack of sufficiently sized experimental datasets, which has constrained machine learning researchers to train primarily on simulated data. However, models trained on simulated pXRD patterns showed limited generalization ability when applied to experimental patterns, particularly for low-quality experimental patterns with high noise levels and elevated backgrounds.
With the Open Experimental Powder X-Ray Diffraction Database (opXRD), we provide an openly available and easily accessible dataset of both labeled and unlabeled experimental powder diffractograms.
%, providing machine learning researchers with a large quantity of real experimental diffractograms collected from a broad range of materials systems
Labeled opXRD data can be used to evaluate the performance of models on real experimental data and unlabeled opXRD data can help improve the performance of models on experimental data, e.g. through transfer learning methods.
We collected XXX labeled and YYY unlabeled diffractograms from a wide spectrum of materials classes, which establishes the first version of the opXRD database.
%We hope that this ongoing effort can guide machine learning research toward fully automated analysis of pXRD data and thus enable future self-driving materials labs.