
\begin{itemize}
    \item Plan for Figures / Tables
    \begin{itemize}
        \item Figure 1: Multiple panels showing histograms of some key metrics: Space group, unit cell volume (normalizable?), ...
        \item Table 1: List of existing powder XRD datasets
        \item Table 2: List of existing ML approaches for pXRD
        \item Table 3: New pXRD datasets offered by this paper
    \end{itemize}

    \item Additional thoughts
    \begin{itemize}
        \item General structure of the paper:
        \begin{itemize}
            \item Introduction
            \item Introduce existing datasets and already show why they are not enough / too small
            \item Summarize SOTA ML that has been applied to experimental pXRD, discuss the limits and gap between performance on simulated data and exp. data (find some good example papers where this is very apparent)
            \item Summary of our dataset, including paragraphs of the different contributions by other groups; comparison with existing datasets, especially in terms of size
        \end{itemize} 
    \end{itemize}
\end{itemize}

% Autonomous laboratory
% https://www.nature.com/articles/s41586-023-06734-w

% Accelerating material discovery
% https://www.nature.com/articles/s41524-022-00765-z

%High-throughput experiments facilitate materials innovation: A review
%https://link.springer.com/article/10.1007/s11431-018-9369-9#:~:text=High%2Dthroughput%20materials%20experiments%20including,at%20a%20fraction%20of%20cost.

The advent of high-throughput experiments, holds the prospect of accelerating the speed materials discovery multiple times over.
The synthesis and probing of novel materials is becoming increasingly efficient and automated,
which has correspondingly increased the throughput of samples in these steps of the experimentation pipeline.

% Rietveld Refinement (3.9 Profile fitting and rietveld analysis)
%https://www.sciencedirect.com/topics/biochemistry-genetics-and-molecular-biology/rietveld-refinement

% Rietveld Refinement: Half a century anniversary
% https://pubs.acs.org/doi/10.1021/acs.cgd.1c00854

% On Rietveld refinement
The objective of powder x-ray diffraction experiments is to extract as much information about the crystal structure
of the sample as possible, or if the sample is multi-phase, also identify the phases present in the sample and quantify
their fractions.
This is done through a technique called Rietveld refinement: A working model of the crystal structure, starting
from an initial guess provided by the operator of the analysis software, is assumed and the corresponding diffractogram
is simulated.
The parameters of the working model of the sample are then adjusted to minimize the difference between the simulated
diffraction pattern and the pattern observed in experiment.
Critically, this process does not generally guarantee to guide the working model towards a global minimum of the difference
in diffractograms and hence arrive at an accurate model of the sample.
In general one can only assume the refinement process to guide the working model towards the nearest local minimum
starting from the initial guess.
Hence, the result of the refinement procedure is in general only as good as the initial guess it was provided with.
Still this workflow often works well, since usually the operator of the experiment will be able
to provide an at least somewhat accurate initial guess of the crystal structure.\\

However, employing manually operated Rietveld Refinement is not without issues.
It is certainly not scalable to the degree that would be required to keep up with advances
in throughput and efficiency in other steps of the experimentation pipeline, even just due to the manual effort involved.
But beyond the lacking scalability of this workflow we also wish to highlight the following issue:
Providing the analysis software with an initial guess of the structure requires manually or algorithmically
consulting a reference library of crystal structures with corresponding simulated or sometimes real diffractograms.
From that reference library a crystal structure must be identified that is likely to be similar
to that of the investigated sample which can then serve as the initial guess provided to the analysis software.
Since the number of chemically viable crystal structures is infinite, any such library will always be incomplete.
The best that can be done to remedy this is to continually update the library both on the side
of the distributor and the user, which is an undesirable overhead.
Especially for novel material discovery, it may be impossible to find a sufficiently accurate match in the reference
library to apply Rietveld Refinement successfully.

Machine learning is a very promising candidate to eliminate both the need for manual inputs and the need for reference
libraries in powder x-ray diffraction analysis, resulting in a streamlined and automated analysis process that can
keep pace with an automated high-throughput experimentation environment. \\

Models can be trained to predict crystal structure properties given only the diffractogram itself, given that
the model can be trained on dataset consisting of pairs of diffractograms and corresponding crystal structures.
So far however, due to an absence of sufficiently large labeled datasets with diffractograms from experiment,
machine learning in this domain has largely relied on simulated diffractograms.
Note, that reference libraries which are used for Rietveld Refinement by in large use simulated diffractograms
rather than diffractograms from experiment and hence cannot either provide a dataset with diffractograms coming
from experiment. \\

Models trained on datasets with simulated diffractograms have already shown very strong performance in predicting
lattice parameters, spacegroup and crystallite size from simulated diffractograms but the performance drops off immensely when these models are applied to data originating from experiment.

The patterns observed in experiment suffer from a host of imperfections that may result in a partial loss of information or make extracting the structural information more difficult. \\
However, there is still reason to believe that the strong performance observed in the domain of purely simulated diffractograms can to a large extent translate to predicting crystal structure properties from diffractograms originating from experiment, given that models are trained on a dataset that accurately reflects diffractograms as they appear in experiment.

However due to the scarcity of labeled diffractograms originating from experiment, collecting a labeled dataset
of real experimental powder x-ray diffraction data of sufficient size to train a model
that can be useful in practice is next to impossible and is not the aim of this dataset.
Rather we hope that this dataset can enable machine learning researchers to evaluate how closely their simulations
represent actual experimental data, identify what features are missing and modify their simulations algorithms accordingly to arrive at a simulation that nearly perfectly mirrors how data appears in reality.\\

There are roughly speaking two paths towards improving the simulation: The first is to more accurately model the sample and the processes taking place within it during the simulation of the powder diffraction pattern. This would for example necessitate modelling absorption of the incoming X-ray radiation and its re-emission in the form of characteristic radiation and modeling the effect of stress and motion within the crystal on the diffractogram. \\
An alternative route to obtaining a more accurate simulation is to try to learn these "imperfections" compared to an idealized simulation using machine learning techniques by extracting them from this dataset. \\
As an example of this could be done, we can investigate results from other areas of research, where a transfer of data content between different "styles" of representation was attempted: \\
Ganin et al.\cite{Ganin2015} proposed a mechanism for training a classifier which performs well independently of which style the input data is represented in, which only requires labeled data from one style and unlabeled data from other styles. This corresponds precisely to the situation at hand, where the number of labeled simulated diffractograms is virtually unlimited and a large amount of unlabeled real diffractograms are available, but labeled real diffractogram are in short suppply. This algorithm could be applied to train a "style-invariant" structure properties prediction model using a combination of our dataset and simulated labeled patterns. \\
Another approach by Gatys et al. \cite{Gatys2016} achieves style transfer of images, by requiring that the content of the output image remain the same, while the style must match that of another image, chosen to represent the target style. This procedure could be applied to transfer the style of simulated diffractorams to that of real diffractograms, which could then be used to train the model data that is more faithful to powder diffractograms as they appear in experiment.


%A first hint of how closely a given simulation reflects experiment could be obtained by
%training a machine learning model, which attempts to classify between simulated and real diffractograms. \\
%If the simulation is very accurate, we would expect the classifier to not have few to no features which it can use to distinguish between simulated and real diffractograms and hence perform poorly. \\
%Conversely, if the classifier performs well and is not difficult to train, that would be an indicator that the simulation doesn't reflect experiment sufficiently well. \\

For a set of given sample properties there is only one corresponding simulated diffractogram, but there may an infinite amount of variations in how the diffractogram of a sample could look like in experiment for example due varying measurement apperatus, noise
, sample impurities and lattice defects.
In order for a simulated dataset to closely resemble experiment it is critical that the simulatio also represents these
statistical variations.
Determining the nature of these variations is a much a harder task than making individual diffractograms look like they
originated from experiment and cannot be accomplished by simply training a classifier.
Nevertheless, we hope that this dataset can at least provide clues of how to take this aspect into account during
simulation as well.\\

Secondly we would like to also present this dataset as a challenge: The fraction of labeled data can be used to easily numerically evalute the performance of the model on a broad range of data as it appears in experiment, which can be used to gauge how well models would perform in practice. \\
But even the unlabeled portion of the dataset can be used to test the performance of models, if the model attempts to predict the entire crystal structure. From the predicted crystal structure, the corresponding diffractogram can be simulated, which can then be compared to the real diffractogram that was used as input for the prediction. While a nearly matching diffractogram doesn't necessarily mean a correctly predicted structure, a non-matching diffractogram certainly means an incorrectly predicted structure. \\


Finally, by inviting the application of models to data as it appears in experiment we would like to bring attention
to some of the practical difficulties that need to be addressed in applying a model to a wide variety of x-ray diffractograms
collected from different sources: For example the angle and the step size of the diffractograms will differ between sources, 
which is something that any model that aims to perform well on this dataset must be equipped to handle.


% TODOS:
% - This paper talks about the importance of sharing raw powder x-ray diffraction data. I'd definitely cite it ->  \cite{Aranda2018}
% - The image diff.png shows the difference between a powder diffractogram calculated from single x-ray diffraction data
% and a real powder diffractogram of the same material. The difference is enormous. I'd include it to highlight
% that simulating powder diffractino patterns from single crystal data only goes so far
%https://www.sciencedirect.com/topics/biochemistry-genetics-and-molecular-biology/rietveld-refinement

%-----------------------------------------------------------%
% Sources on rietveld refinement

% Computationally intensive -> Probably outdated. If it used to take minutes in 2010
% it probably only takes seconds by now.
% https://www.rigaku.com/de/node/707

% A simple solution to the Rietveld refinement recipe problem
% -> Rietveld refinement requires "expert input"
% https://www.ncbi.nlm.nih.gov/pmc/articles/PMC10840306/

% Comments -> Why having to do rietveld refinement sucks
% FullProf tutorial on fitting XRD peaks -> Tedious, requires lots of manual input
% https://pranabdas.github.io/research/fullprof/

% Topas tutorial -> 44 page tutorial, many boring and potentially frustrating steps, lots of points offer potential for task failure
% https://topas.webspace.durham.ac.uk/topas_user_menu/

%  Methods and Tutorials – Powder Diffraction -> Overview of rietveld refinement data
%https://neutrons.ornl.gov/sites/default/files/Methods%20and%20Tutorials%20Powder%20Diffraction.pdf

% Rietveld Refinement from Powder Diffraction Data -> 50 pg tutorial that no one wants to read, overwhelming amout of quantities/figures that needs to be manually looked at in the process
% https://www.fkf.mpg.de/4112052/cpd26.pdf