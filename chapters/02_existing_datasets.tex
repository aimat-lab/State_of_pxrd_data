% Sources to mention
% Zenodo -> A good amount of raw powder data there but spread into smaller datasets; Inhomogenous, not specially prepared to be worked with by others, therefore difficult to work with especially at scale

% Crystal structures with corresponding simulated diffractograms;
% ->  ICSD 
% - 230k structures;
% - Proprietary

% -> PDF  -> 
%- 1200k "material data sets" (structures?) 
% + 19 k entries with "Experimental raw data digital patterns"; 
%  Proprietary (% PDF pricing: https://www.icdd.com/assets/files/2023-Purchase-Options.pdf)
%The PDF itself lists the following data sources:

% ->  COD (https://www.crystallography.net/cod/)
% -  513k structures,
% - Fully open

% Datset listing (Done on: 20.06.24)
% I would list here *all* crystallography databases but evaluate/determine that they lack or don't provide convenient access to raw powder diffraction data

% Some other Lists of Crystallographic databases
% - IUCR: "Crystallographyic databases and related resources" (https://www.iucr.org/resources/data/databases)
% - Bruno et al. (2017): "Crystallography and Databases" (https://datascience.codata.org/articles/10.5334/dsj-2017-038)
% - University of Virginia "Crystallographic data" (https://guides.lib.virginia.edu/c.php?g=514782&p=3568808)

%->1 PDF5+ (https://www.icdd.com/,https://www.icdd.com/pdf-5/)
% - Data sources of the PDF
% (1.1): ICDD Powders
% (1.2): FIZ ICSD (https://icsd.products.fiz-karlsruhe.de/)
% (1.3): NIST ICSD https://icsd.nist.gov/
% (1.4): Linus Pauling File (https://paulingfile.com/index.php?p=about%20us)
% (1.5): ICDD Single Crystal data
%-> 2 COD (https://www.crystallography.net/cod/)
%-> 3 RRUFF (American Mineralogist Crystal Structure Database)
%-> 4 CCDC (https://www.ccdc.cam.ac.uk/)
%-> 5 The Material Project (https://legacy.materialsproject.org/)
%-> 6 Crystal Lattice Structures (https://www.atomic-scale-physics.de/lattice/)
%-> 7 Crystallographic and Crystallochemical Database for Minerals and their Structural Analogues (https://database.iem.ac.ru/mincryst/)
%-> 8 Mineralogy Database (https://webmineral.com/)
%-> 9 IUCr Raw data letters (https://iucrdata.iucr.org/)
%-> 10 Bilbao Crystallographic server  (https://www.cryst.ehu.es/)
%-> 11 PowBase (http://www.cristal.org/powbase/index.html)
%-> 12 Athena (https://athena.unige.ch/athena/mineral/mineral.html)
%-> 13 Protein database (https://www.rcsb.org/)
%-> 14 SDPD (Structure Determination from Powder Diffraction - Database)
%-> 15 Crystal Met (https://cds.dl.ac.uk/cds/datasets/crys/mdf/llmdf.html)
%-> 16 NAKB (https://www.nakb.org/)
%-> 17 Pearson's Crystal data (https://www.crystalimpact.com/pcd/)
%-> 18 Biological macromolecule Crystallization Database (http://bmcd.ibbr.umd.edu/)

%TODO: Also include plans for NOMAD and highlight their efforts

%TODO: Summarize the existing datasets shortly, including number of entries, etc.

\paragraph{Existing powder diffraction datasets} To contextualize opXRD within the current environment of experimental powder diffraction data, the following list provides an overview of the largest crystal structure databases that include powder diffraction data. 

\begin{itemize}
\item \textbf{Powder Diffraction File:} \footnote{More information on the Powder Diffraction File can be found under \url{https://www.icdd.com/pdf-5/}.} The Powder Diffraction File (PDF), published and maintained by the International Center for diffraction data, is currently the largest collection of powder diffraction data available to researchers. Though its latest release contains over a million entries, almost all are simulated patterns. Only 19000 entries represent powder diffraction data originating from experiments, and of those, only about 11000 come with accompanying atomic coordinates. It is widely used as a reference database in Rietveld refinement to identify crystal structures from which the refinement process can begin based on their powder diffraction patterns. As of November 2024, the latest release of the PDF, the PDF5+, is available to researchers through a purchase starting at a price point of \$6,265.00.

\item \textbf{Crystallography Open Database:}\footnote{More information on the crystallography open database can be found under \url{https://www.crystallography.net/cod/}.} The Crystallography Open Database (COD)\cite{Graulis2009cod} is an open-access collection of crystal structures. It provides over 500.000 crystal structures in the form of .cif files. Of these files, 1063 contained the experimental powder diffraction data that was used to determine the underlying crystal structures of the investigated samples.

\item Scattered experimental datasets on Zenodo (TODO: roughly estimate size)
\item \textbf{RRUFF}: The RRUFF Mineral Database \cite{Armbruster2015} provides detailed information on minerals, including their chemical compositions, crystallography, and spectroscopic data. Managed by the University of Arizona, it was created to serve as a public repository for mineral identification and research. It contains \num{908} distinct structures with experimental x-ray diffraction patterns.
\item ...
\end{itemize}

Some major crystal structure databases, such as the ICSD, do not appear in the above list, as they do not include powder diffraction data. This is because most structure solutions are achieved through single-crystal diffraction rather than powder diffraction.
 
