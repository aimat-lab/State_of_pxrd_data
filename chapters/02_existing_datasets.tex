% Datset listing (List written on 20.06.24)
% Lists of Crystallographic databases; Our list contains all of the databases given in the lists linked below
% - IUCR: "Crystallographic databases and related resources" (https://www.iucr.org/resources/data/databases)
% - Bruno et al. (2017): "Crystallography and Databases" (https://datascience.codata.org/articles/10.5334/dsj-2017-038)
% - University of Virginia "Crystallographic data" (https://guides.lib.virginia.edu/c.php?g=514782&p=3568808)

% [X] 1 PDF5+ (https://www.icdd.com/,https://www.icdd.com/pdf-5/) -> 19000 structures with pXRD patterns | I checked with their support
    % (1.1): ICDD Powders
    % (1.2): FIZ ICSD (https://icsd.products.fiz-karlsruhe.de/)
    % (1.3): NIST ICSD https://icsd.nist.gov/
    % (1.4): Linus Pauling File (https://paulingfile.com/index.php?p=about%20us)
    % (1.5): ICDD Single Crystal data
% [x]: 2 COD (https://www.crystallography.net/cod/) -> ~ 1000 structures with real pXRD patterns | These patterns were extracted by PSL university
% [x]: 3 RRUFF (American Mineralogist Crystal Structure Database) -> ~ 1000 structures with real pXRD patterns | We downloaded the entire dataset and discarded single crystal data
% [x]: 4 CCDC (https://www.ccdc.cam.ac.uk/) -> No exp pXRD | There is only mention of simulated powder diffraction data. Of 20 sampled cifs none contained powder diffraction data
% [x]: 5 The Material Project (https://legacy.materialsproject.org/) -> No exp pXRD | The only mention of available X-ray diffraction data is of simulated data
% [x]: 6 Crystal Lattice Structures (https://www.atomic-scale-physics.de/lattice/) -> No exp pXRD | Their entries provide no X-ray diffraction data, and also most links appear broken
% [x]: 7 Crystallographic and Crystallochemical Database for Minerals and their Structural Analogues (https://database.iem.ac.ru/mincryst/) -> No exp pXRD | The given powder diffraction data is simulated (see: Powder X-ray Diffraction Standards (CPDS) @ https://database.iem.ac.ru/mincryst/descript.htm)
% [x]: 8 Mineralogy Database (https://webmineral.com/) -> No exp pXRD | I checked with the author of the site
% [x]: 9 IUCr Raw data letters (https://iucrdata.iucr.org/) -> No exp pXRD | There are only two raw data letters, both of them single crystal data
% [x]: 10 Bilbao Crystallographic server  (https://www.cryst.ehu.eus/bincstrdb/search/) No exp pXRD | Of 20 sampled CIFs out of 256 entries none contained pXRD data
% [x]: 11 PowBase (http://www.cristal.org/powbase/index.html) -> 169 real pXRD patterns some of which have at least partial labels; The described submission process and the lack of crystal structures indicate that the patterns are not simulated
% [x]: 12 Athena (https://athena.unige.ch/athena/mineral/mineral.html) -> No exp pXRD | No mention of diffraction data anywhere
% [x]: 13 Protein database (https://www.rcsb.org/) / NAKB (https://www.nakb.org/) / BMCD ((http://bmcd.ibbr.umd.edu/)  -> Little to no exp pXRD | Of the 230k structures only 21 showed up when searching experimental method = Powder diffraction
% [x]: 14 Crystal Met (https://cds.dl.ac.uk/cds/datasets/crys/mdf/llmdf.html) -> No exp pXRD | The only mention of diffraction data is simulated data
% [x]: 15 Pearson's Crystal data (https://www.crystalimpact.com/pcd/) -> 21700 experimental powder diffraction patterns (As per https://www.crystalimpact.com/pcd/) | The Pearson's crystal data evolved from the Linus Pauling File which is a data source of the PDF, so there may be significant overlap in these patterns
% [x]: 16 Zenodo powder diffraction data (https://zenodo.org/search?q=powder%20diffraction&l=list&p=1&s=10&sort=bestmatch) -> Some pXRD data but hard to quantify, probably <= 4000 patterns most of which are unlabeled | 1600 datasets matching "powder diffraction". From probing among the first 100 matches I estimate only about 1 in 10 contain powder diffraction patterns. The probed datasets that did contain powder diffraction data only contained <= 30 patterns and only one of them came with corresponding crystal structures

%TODO: Summarize the existing datasets shortly, including number of entries, etc.

\pagebreak

\subsubsection*{Existing experimental powder diffraction datasets}
To contextualize opXRD within the current environment of experimental powder diffraction data, the following list provides an overview of the largest crystal structure databases that include powder diffraction data.  \\

\textbf{Powder Diffraction File:}\footnote{More information on the Powder Diffraction File can be found under \url{https://www.icdd.com/pdf-5/}.} The Powder Diffraction File (PDF), published and maintained by the International Center for Diffraction Data (ICDD), is a large collection of materials with accompanying powder diffraction data. Its latest release as of November 2024, the PDF5+, contains over a million materials with accompanying powder diffraction data. However, most of these powder diffraction patterns are simulated. After inquiring with the ICDD in April 2024 we were told that only 19000 of the powder diffraction patterns in the PDF5+ stemmed from experiments. Of these 19000 entries, 10954 contain information about the atomic coordinates of the underlying structures. \footnote{We reached out to the contact address given on their website, info@icdd.com.} Still, this makes the PDF5+, to the best of our knowledge, the largest collection of experimental powder diffraction data available to researchers. As of November 2024, the PDF5+ is available to researchers through a purchase starting at a price point of \$6,265.00.\\

\textbf{Crystallography Open Database:}\footnote{More information on the crystallography open database can be found under \url{https://www.crystallography.net/cod/}.} The Crystallography Open Database (COD)\cite{Graulis2009cod} is an open-access collection of crystal structures. It provides over 500.000 crystal structures in the form of .cif files. Of these files, 1063 contained the experimental powder diffraction data that was used to determine the underlying crystal structures of the investigated samples. Hence the experimental powder diffraction data contained in the COD is fully labeled. The .cif files include both lattice parameters and the contents of the unit cell of the underlying structure. \\

\textbf{RRUFF}: The RRUFF Mineral Database \cite{Armbruster2015} provides detailed information on minerals, including their chemical compositions, crystallography, and spectroscopic data. Managed by the University of Arizona, it was created to serve as a public repository for mineral identification and research. It contains \num{1290} powder diffraction patterns stemming from experiments each labeled with the lattice parameters and composition of the underlying structures.

\textbf{PowBase}: 

\textbf{Pearson's Crystal data}:

Some major crystal structure databases, such as the ICSD, do not appear in the above list, as they do not include powder diffraction data. This is because most structure solutions are achieved through single-crystal diffraction rather than powder diffraction.
 
