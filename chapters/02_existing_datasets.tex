\pagebreak

\subsubsection*{Existing experimental powder diffraction databases}
To contextualize opXRD within the current environment of experimental powder diffraction data, the following list provides an overview of the largest crystal structure databases that offer access to experimental powder diffraction data. For an overview of these databases refer to Tab.$\eqref{tab:exp_databases}$ below.\\

\begin{table}[!htb]
\centering
\caption{Overview of Experimental Powder Diffraction Databases}
\label{tab:exp_databases}
\begin{tabular}{@{}lcc@{}}
\toprule
\textbf{Name}                          & \textbf{No. of Patterns} & \textbf{Year Established} \\
\midrule
Powder Diffraction File (PDF)          & 19,000                  & 1941                      \\
Pearson's Crystal Data                 & 21,700                  & 2007        \\
Crystallography Open Database (COD)    & 1063                    & 2003                      \\
RRUFF                                  & 1290                    & 2006                      \\
PowBase                                & 169                     & 1999                      \\
\bottomrule
\end{tabular}
\end{table}

\textbf{Powder Diffraction File:}\footnote{More information on the Powder Diffraction File can be found under \url{https://www.icdd.com/pdf-5/}.} The Powder Diffraction File (PDF), published and maintained by the International Center for Diffraction Data (ICDD), is a large collection of materials with accompanying powder diffraction data first published in 1941\cite{GatesRector2019}. According to the ICDD the PDF5+, the latest release of the PDF as of November 2024, contains over a million materials with accompanying powder diffraction data. However, most of these powder diffraction patterns are simulated.\footnote{This information is available under \url{https://www.icdd.com/pdf-5/}.} After inquiring with the ICDD in April 2024 we were told that only 19,000 of the powder diffraction patterns in the PDF5+ stemmed from experiments. Of these 19,000 entries, 10,954 contain information about the atomic coordinates of the underlying structures.\footnote{We reached out to the contact address given on their website, info@icdd.com.} Still, this makes the PDF5+, to the best of our knowledge, the largest collection of experimental powder diffraction data available to researchers. As of November 2024, the PDF5+ is available to researchers through a purchase of a one-year license starting at a price point of \$6265.00.\\

\textbf{Pearson's Crystal data}:\footnote{More information on Pearson's crystal data can be found on the ASM International website under \url{https://www.asminternational.org/materials-resources/online-databases/pearsons-crystal-data-crystal-structure-database-for-inorganic-compounds/?srsltid=AfmBOorjpwJSVNsaWtCO4kmgu-Ox2ncF3tkzwCXRE2GTemYIxsSLp8xr}.} Pearson's crystal data is a commercial database of 395,000 crystal structures including 21,700 experimental powder diffraction patterns maintained by ASM International first published in 2003.\footnote{According to CrystalImpact, who acts as distributor for Pearson's crystal data. For more information refer to the CrystalImpact website under \url{https://web.archive.org/web/20241112132017/https://www.crystalimpact.com/news/Default.htm}. } Pearson's crystal data is based on the data of another crystal structure database, the Linus Pauling file \cite{kaduk2007}. Since the Linus Pauling file is also a database source for the PDF, there is likely a significant overlap between the experimental powder diffraction data found in Pearson's crystal data and that found in the PDF. As of November 2024, Pearson's crystal data is available to researchers through a purchase of a one-year license starting at a price point of 2200 \euro.\\

\textbf{Crystallography Open Database:}\footnote{More information on the crystallography open database can be found under \url{https://www.crystallography.net/cod/}.} The Crystallography Open Database (COD) is an open-access collection of crystal structures founded in 2003\cite{Graulis2009cod}. It currently provides over 500,000 crystal structures in the form of .cif files. Of these files, 1063 contain the experimental powder diffraction data that was used to determine the underlying crystal structures of the investigated samples. Hence the experimental powder diffraction data contained in the COD is fully labeled. The .cif files include both lattice parameters and the contents of the unit cell of the underlying structure. \\

\textbf{RRUFF}:\footnote{More information on RRUFF can be found under \url{https://rruff.info/about/about_general.php}.} The RRUFF Mineral Database, first published in 2006, provides detailed information on minerals, including their chemical compositions, crystallography, and spectroscopic data. Managed by the University of Arizona, it was created to serve as a public repository for mineral identification and research. It contains \num{1290} powder diffraction patterns stemming from experiments each labeled with the lattice parameters and composition of the underlying structures. \\

\textbf{PowBase}:\footnote{More information on PowBase can be found under \url{http://www.cristal.org/powbase/index.html}. PowBase was an initiative suggested in the Structure Determination by
Powder Diffractometry (SDPD) mailing list which was hosted on the same site. The COD was another community initiative that grew out of this mailing list.} PowBase is a database of 169 mostly unlabeled experimental powder diffraction patterns collected and maintained by crystallography researcher Armel Le Bail starting in 1999. As of November 2024, all 169 patterns are still freely available for download. \\


There is also publicly available powder diffraction data uploaded to datasets on Zenodo\footnote{More information on Zenodo can be found under \url{https://zenodo.org/}.}. However, this data is split into disparate entries that typically only contain the work of a single research project. Additionally, extracting powder diffraction data at scale is hindered by the fact that the data is often given in plain text files in non-standardized formats, which are difficult to parse programmatically. \\

Aside from the databases mentioned above, we have also investigated several other crystal structure resources in search of experimental powder diffraction data. Crystal structure resources that were investigated but not found to contain any appreciable amount of publicly available experimental powder diffraction data include the Inorganic Crystal Structure Database,\footnote{More information on the ICSD can be found under \url{https://icsd.products.fiz-karlsruhe.de/}.}   Cambridge Structural Database,\footnote{More information on the CCDC can be found under \url{https://www.ccdc.cam.ac.uk/structures/}.}  the Materials Project database,\footnote{More information on the Materials Project can be found under\url{https://next-gen.materialsproject.org/}.}  the Crystallographic and Crystallochemical Database\footnote{More information on the Crystallographic and Crystallochemical Database can be found under \url{https://database.iem.ac.ru/mincryst/index.php}.},  the Bilbao Incommensurate Crystal Structure Database,\footnote{More information on the Incommensurate Crystal Structure Database can be found under \url{https://www.cryst.ehu.eus/bincstrdb/search/}.}  the Mineralogy Database,\footnote{More information on the Mineralogy database can be found under \url{https://webmineral.com/}.} IUCr Raw data letters,\footnote{More information on the IUCr Raw Data letters can be found under \url{https://iucrdata.iucr.org/x/index.html}.} and U.S. Naval Research Laboratory Crystal Lattice-Structures,\footnote{More information on the Crystal Lattice-Structures database can be found under \url{https://www.atomic-scale-physics.de/lattice/}.} the Athena Mineral database \footnote{More information on the Athena Mineral database can be found under \url{https://athena.unige.ch/athena/mineral/mineral.html}.} and the Protein data bank. \footnote{More information on the Protein data bank can be found under \url{https://www.rcsb.org/}.} This is to be expected as most structure solutions are achieved through single-crystal diffraction rather than powder diffraction.\\

