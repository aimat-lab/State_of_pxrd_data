% Sources to mention
% Zenodo -> A good amount of raw powder data there but spread into smaller datasets; Inhomogenous, not specially prepared to be worked with by others, therefore difficult to work with especially at scale

% Crystal structures with corresponding simulated diffractograms;
% ->  ICSD 
% - 230k structures;
% - Proprietary

% -> PDF  -> 
%- 1200k "material data sets" (structures?) 
% + 19 k entries with "Experimental raw data digital patterns"; 
%  Proprietary (% PDF pricing: https://www.icdd.com/assets/files/2023-Purchase-Options.pdf)
%The PDF itself lists the following data sources:

% ->  COD (https://www.crystallography.net/cod/)
% -  513k structures,
% - Fully open

% Datset listing (Done on: 20.06.24)
% I would list here *all* crystallography databases but evaluate/determine that they lack or don't provide convenient access to raw powder diffraction data

% Some other Lists of Crystallographic databases
% - IUCR: "Crystallographyic databases and related resources" (https://www.iucr.org/resources/data/databases)
% - Bruno et al. (2017): "Crystallography and Databases" (https://datascience.codata.org/articles/10.5334/dsj-2017-038)
% - University of Virginia "Crystallographic data" (https://guides.lib.virginia.edu/c.php?g=514782&p=3568808)

%->1 PDF5+ (https://www.icdd.com/,https://www.icdd.com/pdf-5/)
% - Data sources of the PDF
% (1.1): ICDD Powders
% (1.2): FIZ ICSD (https://icsd.products.fiz-karlsruhe.de/)
% (1.3): NIST ICSD https://icsd.nist.gov/
% (1.4): Linus Pauling File (https://paulingfile.com/index.php?p=about%20us)
% (1.5): ICDD Single Crystal data
%-> 2 COD (https://www.crystallography.net/cod/)
%-> 3 RRUFF (American Mineralogist Crystal Structure Database)
%-> 4 CCDC (https://www.ccdc.cam.ac.uk/)
%-> 5 The Material Project (https://legacy.materialsproject.org/)
%-> 6 Crystal Lattice Structures (https://www.atomic-scale-physics.de/lattice/)
%-> 7 Crystallographic and Crystallochemical Database for Minerals and their Structural Analogues (https://database.iem.ac.ru/mincryst/)
%-> 8 Mineralogy Database (https://webmineral.com/)
%-> 9 IUCr Raw data letters (https://iucrdata.iucr.org/)
%-> 10 Bilbao Crystallographic server  (https://www.cryst.ehu.es/)
%-> 11 PowBase (http://www.cristal.org/powbase/index.html)
%-> 12 Athena (https://athena.unige.ch/athena/mineral/mineral.html)
%-> 13 Protein database (https://www.rcsb.org/)
%-> 14 SDPD (Structure Determination from Powder Diffraction - Database)
%-> 15 Crystal Met (https://cds.dl.ac.uk/cds/datasets/crys/mdf/llmdf.html)
%-> 16 NAKB (https://www.nakb.org/)
%-> 17 Pearson's Crystal data (https://www.crystalimpact.com/pcd/)
%-> 18 Biological macromolecule Crystallization Database (http://bmcd.ibbr.umd.edu/)

%TODO: Also include plans for NOMAD and highlight their efforts

\paragraph{Existing datasets} TODO: Summarize the existing datasets shortly, including number of entries, etc.

\begin{itemize}
\item RRUFF
\item Powder Diffraction File (ICDD)

\item Scattered experimental datasets on Zenodo (TODO: roughly estimate size)
\item ...
\end{itemize}
 
