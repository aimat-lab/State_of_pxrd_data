
% TODOS:
% - This paper talks about the importance of sharing raw powder X-ray diffraction data. I'd definitely cite it ->  \cite{Aranda2018}

%- The subsections of this section should be the 
% individual contributions that we get from labs

%------------------------------------------$

\begin{figure*}[!ht]
    \centering
    \missingfigure{} 
    \caption{Statistics, histograms, etc. of our dataset.}
    \label{fig:statistics}
\end{figure*}

In collaboration with several other institutions we have collected a dataset of diffractograms stemming from experiment, some of them labeled with corresponding structural information. 

\begin{itemize}
    \item Institute of Nanotechnology at Karlsruhe Institute of Technology
    \item University of Southern California
    \item Lawrence Berkeley National Laboratory
    \item Hong Kong University of Science and Technology (Guangzhou)
    \item Swiss Federal Laboratories for Materials Science and Technology
\end{itemize}

%% Paragraph by FX Coudert and Arthur Hardiagon
\subsection*{Chimie Paris Tech, PSL University}

We extracted experimental pXRD data from the Crystallography Open Database (COD).\cite{Grazulis2009, Vaitkus2023} The COD is, to our knowledge, the largest open-access collection of experimental crystal structures of organic, inorganic, and metal-organic compounds and minerals, containing more than 500,000 entries. \footnote{Available online at \url{https://www.crystallography.net/cod/}.} The data in the COD is placed in the public domain and licensed under the CC0 License. Of the entire COD database, 5432 structures 1\% of the database) contained at least one tag from the {CIF\_POW} dictionary, i.e., relating to powder diffraction studies --- which is expected, since most crystal structures are resolved from single crystal diffraction. Of these 5432 files, most contained only metadata related to the powder diffraction experiment, but did not include the raw data of the pattern itself. We could extract raw, experimental pXRD patterns from 1063 files in total, after curation of a small number of files with clearly invalid data. \\

The pXRD data from the COD database are of high quality, with a median resolution of $\Delta(2\theta) = 0.013${\textdegree} and an average number of 9190 points measured per pattern. They span a wide chemical space, including organic, inorganic, and hybrid structures, including 75 different elements of the periodic table.

%% Paragraph by Ben Breitung
\subsection*{Institute of Nanotechnology, KIT}

The X-ray diffraction data were collected on a Bruker D8 Advance (Cu K$\alpha$1 radiation, $\lambda$ = 1.54056 Å). The samples were initially recorded for various research projects over the last ten years and were measured with different step sizes, times per step, and over different angle ranges, but all using Cu K$\alpha$1 radiation. The samples mostly contained transition metal oxides, sulfides, and fluorides. A major part of the research focused on high-entropy materials, which involved incorporating many different elements into single-phase structures, leading to peak shifts or phase separations. Most of those multi-component complex materials appeared in various structures, including rock-salt, spinel, fluorite, perovskite, and delafossite, to name a few.The samples were almost always prepared in powder form; therefore, powder XRD was performed on samples with adjusted height. To improve statistics, the samples were rotated during the entire measurement. Some air-sensitive samples were measured using a transparent polymer dome for protection. This dome led to increased background noise over the first 20° and slightly decreased pattern resolution. The samples were prepared using various synthesis techniques, mostly solid-state or wet chemical syntheses, to obtain the desired structures. Consequently, particle size and crystallinity varied significantly. The sample set also includes samples that were not successfully measured or where phases could not be identified. 

% Paragraph by Tim Kodalle
\subsection*{Lawrence Berkeley National Lab}

Experimental XRD data were collected at beamline 12.3.2 of the Advanced Light Source, the synchrotron at Lawrence Berkeley National Laboratory, using a photon energy of 10 keV (corresponding to a wavelength of $\lambda$ = 1.23984193 Å), selected using a Si(111) monochromator. Measurements were taken in grazing incidence geometry, i.e. using a beam incidence angle of 1$^{\circ}$. Two-dimensional diffraction images were recorded using a Dectris Pilatus 1M area detector at an angle between 34$^{\circ}$ and 36$^{\circ}$ with a sample-to-detector distance of roughly 190 mm. The two-dimensional data were calibrated using an Al$_{2}$O$_{3}$ calibration standard and integrated along the azimuthal angle. Data collection was performed in situ during thin-film deposition using a custom-made spin-coating and annealing stage. One dataset was collected spin-coating triple cation metal-halide perovskite precursor solutions with the composition Cs$_{0.05}$(MA$_{0.23}$FA$_{0.77}$)Pb$_{1.1}$(I$_{0.77}$Br$_{0.23}$)$_{3}$ (MA = Methylammonium, FA = Formamidinium) onto various substrates, including glass (amorphous), GaAs wafers (single crystalline) as well as stacks of glass/ITO, GaAs/Mo/Cu(In,Ga)Se$_{2}$/CdS/ZnO, and glass/Mo/Cu(In,Ga)Se$_{2}$/CdS/ZnO (glass/CIGS). Some of the substrates were additionally covered with a self-assembling monolayer of MeO-2PACz. The GaAs substrates were prepared by Dr. Jiro Nishinaga from the National Institute of Advanced Industrial Science and Technology (AIST) in Japan and the glass/CIGS substrates by Dr. Christian Kaufmann and his team at Helmholtz-Zentrum Berlin (HZB) in Germany. \\

A second dataset was collected from spin-coating metal-halide perovskite precursor solutions with varying compositions of \ce{MAPb(I_{1-x}Br_x)3} (where MA = Methylammonium and x = 0, 0.33, 0.5, 0.67, 1) spin-coated onto glass substrates. The substrates were preheated to different temperatures (30°C, 50°C, 70°C, and 90°C), and the spin-coating process was performed at a constant temperature on the preheated substrates. For both datasets, diffraction data were continuously measured during spin-coating, chemical induction of crystallization, and annealing of the samples (at 100$^{\circ}$C and 110$^{\circ}$C respectively) with a frequency of about 0.56 1/s and  0.54 1/s. Each in situ measurement consisted of about 500 to 1000 individual diffractograms. Depending on the substrate, each series of diffractograms shows an evolution from substrate only to a combination of polycrystalline perovskite, PbI$_{2}$ and substrate via several intermediate phases.


% Paragraph by Bin Cao
\subsection*{Guangzhou Municipal Key Laboratory of Materials Informatics, HKUST}

In the past two years, we established a small-scale experimental PXRD database called the X-Ray Phase Identification Public Experimental Dataset (XRed,  \href{https://github.com/WPEM/XRED}{https://github.com/WPEM/XRED}). The data in XRed is generated in our lab using instruments such as the Empyrean 3.0, Aeris, and Bruker D8 Advance diffractometers under copper targets. Hundreds of PXRD patterns have been refined, labeled with CIF files, and organized by elemental systems. The dataset includes original experimental files across single-phase to five-phase mixtures. This project is progressing toward integration with the comprehensive opXRD database to establish a large-scale, long-term experimental resource. 

% Paragraph by Moritz Wolf and Tony-yi 
\subsection*{Institute of Catalysis Research and Technology, KIT}

X-ray diffraction (XRD) was conducted with an X’Pert Pro MPD (Panalytical) in Bragg-Brentano geometry using a copper X-ray source ($\lambda_{K_{\alpha}} = 1.5406 Å$). The patterns were acquired from 5-80° $2\theta$ with a step size of 0.016711° or 0.033420° and a total acquisition time of 40 to 120 min. A variety of samples were analyzed including commercial catalysts, bulk reference materials, porous metal oxide particles, and nanoparticles. The latter were synthesized via the surfactant-free benzyl alcohol route (10.1039/C9DT01634A, 10.1016/j.matchemphys.2018.04.021). The cobalt oxide (CoO or Co3O4) and cerium oxide (CeO2) nanoparticles were in the size range of 4-16 nm according to the Scherrer equation. A series of porous Al2O3 materials, which were prepared by calcination of boehmite (AlOOH) at various temperatures, represents crystalline samples with limited long-range structure and various contributions of Al2O3 polymorphs.

% Paragraph by Alexander Wieczorek and Sebastiann Siol
\subsection*{Laboratory for Surface Science and Coating Technologies, Empa}

XRD data was measured using a Bruker D8 Discover equipped with a Cu~K$\alpha$ radiation source ($\lambda$~=~1.54056~{\AA} for Cu~K$\alpha$\textsubscript{1} and $\lambda$~=~1.54439~{\AA} for Cu~K$\alpha$\textsubscript{2}) in a Bragg-Brentano geometry. For the reported data sets the instrument was equipped with a Goebel mirror effectively removing the Cu~K$\beta$ radiation. The data set originates from the combinatorial exploration of the Zn–V–N compositional space, as well as data gathered from multiple research activities on more established metal halide perovskite semiconductors. All data was collected from thin films deposited on borosilicate glass. The Zn–V–N films showed some preferential out of plane orientation, while for the perovskites the preferential orientation was minimal, resulting in the presence of all reflections.
Each combinatorial Zn–V–N library was synthesized using radio-frequency co-sputtering of Zn and V in a mixed Ar and N\textsubscript{2} plasma. An orthogonal deposition temperature and composition gradient was created, resulting in a deposition temperature of 220~{\textdegree}C for samples 1~–~9 and 114~{\textdegree}C for samples 37~–~45. The composition for each sample was determined using X-Ray fluorescence (XRF) spectroscopy which was further calibrated through Rutherford backscattering spectroscopy (RBS) based on selected samples. The newly identified and isolated semiconductor Zn\textsubscript{2}VN\textsubscript{3} was identified to exhibit a cation-disordered wurtzite structure as verified by additional GI-XRD and SAED measurements.\cite{Zhuk2021}
Tin halide perovskites were deposited using single-step spin-coating as reported elsewhere.\cite{Wieczorek2023} Methylammonium lead iodide libraries with varying degrees of residual PbI\textsubscript{2} were deposited using a two-step procedure involving both thermal evaporation of PbI\textsubscript{2} and subsequent spin-coating of a methylammonium solution. The relative phase fractions were quantified using supplementary azimuthal angle scans coupled with structural factors and geometrical factors as reported elsewhere.\cite{Wieczorek2024} Fully inorganic lead perovskite libraries were prepared using thermal co-evaporation of lead and cesium halide salts.
All metal halide perovskite libraries were measured within a custom-made X-Ray transparent inert-gas dome, resulting in the presence of minor additional features within the $\theta$~=~19~–~31{\textdegree} range. For all combinatorial libraries where any phases are specified, the complete set of phases are reported in the meta data.

We are still in the process of growing the dataset and additional contributions are still very welcome. To find out more about how to contribute to this dataset, visit our website specially designed to collect this dataset, \url{https://xrd.aimat.science} .