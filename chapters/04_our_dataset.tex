
% TODOS:
% - This paper talks about the importance of sharing raw powder X-ray diffraction data. I'd definitely cite it ->  \cite{Aranda2018}

%- The subsections of this section should be the 
% individual contributions that we get from labs

%------------------------------------------$

\begin{figure*}[!ht]
    \centering
    \missingfigure{} 
    \caption{Statistics, histograms, etc. of our dataset.}
    \label{fig:statistics}
\end{figure*}

In collaboration with several other institutions we have collected a dataset of diffractograms stemming from experiment, some of them labeled with corresponding structural information. \\
Currently, institutions that contributed to our dataset include:
\begin{itemize}
    \item Institute of Nanotechnology at Karlsruhe Institute of Technology
    \item University of Southern California
    \item Lawrence Berkeley National Laboratory
\end{itemize}

%% Paragraph by FX Coudert and Arthur Hardiagon

We extracted experimental pXRD data from the Crystallography Open Database (COD).\cite{Grazulis2009, Vaitkus2023}  The COD is, to our knowledge, the largest open-access collection of experimental crystal structures of organic, inorganic, metal-organic compounds and minerals, containing more than 500,000 entries.\footnote{Available online at \url{https://www.crystallography.net/cod/}} The data in the COD is placed in the public domain and licensed under the CC0 License. Of the entire COD database, 5432 structures (1\% of the database) contained at least one tag from the {CIF\_POW} dictionary, i.e., relating to powder diffraction studies --- which is expected, since most crystal structures are resolved from single crystal diffraction. From these 5432 files, most contained only metadata related to the powder diffraction experiment, but did not include the raw data of the pattern itself. We could extract raw, experimental pXRD patterns from 1063 files in total, after curation of a small number of files with clearly invalid data.

The pXRD data from the COD database are of high quality, with a median resolution of $\Delta(2\theta) = 0.013$\textdegree and an average number of 9190 points measured per pattern. They span a wide chemical space, including organic, inorganic and hybrid structures, including 75 different elements of the periodic table.

%%

We are still in the process of growing the dataset and additional contributions are still very welcome. To find out more about how to contribute to this dataset, visit our website specially designed to collect this dataset, https://xrd.aimat.science .