\subsubsection*{How to contribute and how to use}

TODO: A small tutorial with a few python commands to show how easy it is to use our dataset through the provided API.

A major advantage of this dataset over other comparable datasets is that it is very easy to handle in python and in PyTorch.
Our accompanying python library xrdpattern (\url{https://github.com/aimat-lab/xrdpattern}) provides a means to read the dataset simply by using PatternDB.load(data\_dirpath). The patterns attribute of this class is a list of the indivdual pattern diffractograms, each of which supports a standardize, plot and to tensor method. \\
The standardize method returns a "standardized" version of the pattern with a fixed angle range, a fixed number of entries and intensities normalized to the $[0,1]$ interval.

% 
We have launched Xqueryer (\url{http://xqueryer.caobin.asia/}), a cutting-edge, free online platform for intelligent structure identification. Users can upload experimental pXRD data to identify crystal structures and extract all relevant information available in the Materials Project database. All usage complies with Xqueryer's terms of service. Uploaded data are securely stored and integrated into the opXRD database.