Powder X-ray diffraction (pXRD) experiments are a cornerstone for materials structure characterization.
Despite its widespread application, the analysis of the diffractograms still presents a significant bottleneck
in high-throughput experimentation that remains to be solved in order to match the increased throughput
of the prior stages of experimentation. In the current landscape of powder X-ray diffraction analysis, machine learning has emerged as a promising research direction to resolve this bottleneck by enabling automated powder diffraction analysis.
A notable difficulty in applying machine learning to this domain is the lack of labeled datasets, i.e.\
diffraction patterns originating from experiment with accompanying structure solutions, which has relegated machine learning
researchers to train on simulated data.
Since these simulations largely fail to accurately reflect the experiment, the performance of models trained
on this simulated data by and large fails to transfer to experiment and provide value in practice.
We aim to remedy this by providing a freely available and easily accessible dataset of partially labeled
powder diffraction data, providing machine learning researchers with a large quantity of real experimental diffractograms collected from a broad range of samples.
We hope that this can guide machine learning research toward more realistic handling of experimental imperfections to enable the improvement of current automated analysis workflows.