In powder x-ray diffraction, the analysis of the diffractograms still presents a significant bottleneck
in high-throughput experimentation that remains to be solved, in order to match the increased throughput
of the prior stages of experimentation.
While Rietveld refinement can identify the phases and extract the structural information found
in the diffractograms, each sample requires manual inputs and careful configuration of the analysis software
from trained staff, which is infeasible for highly automated high-throughput experiments.

In the current landscape of powder x-ray diffraction analysis, machine learning has emerged as a very promising direction of research resolving this bottleneck by enabling fully automated powder diffraction analysis.
A noteable difficulty in applying machine learning to this domain is the lack of labeled datasets i.e.\
diffraction patterns originating from experiment with accompanying structure solutions, which has relegated machine learning
researchers to training on simulated data.
Since these simulations largely fail to accurately reflect experiment, the performance of models trained
on this simulated data, also by in large fails to transfer to experiment and provide value in practice.
We aim to remedy this by providing a freely available and easily accessible dataset of partially labeled
powder diffraction data, providing machine learning researchers with a large quantity of real experimental diffractograms
collected from a broad range of samples.
We hope that this can guide machine learning research towards more realistic
simulations and potentially enable training and testing models on real experimental data.



%Machine learning techniques have successfully been used to extract structural information such as the crystal space group from powder X-ray diffractograms.
%However, training directly on simulated diffractograms from databases such as
%the ICSD is challenging due to its limited size, class-inhomogeneity, and bias
%toward certain structure types. We propose an alternative approach of generating
%synthetic crystals with random coordinates by using the symmetry operations of
%each space group. Based on this approach, we 
%demonstrate online training of deep ResNet-like models on up to a few
%million unique on-the-fly generated synthetic diffractograms per hour. For our chosen task of
%space group classification, we achieved a test accuracy of 79.9\% on unseen ICSD
%structure types from most space groups. This surpasses the 56.1\% accuracy of the current state-of-the-art approach of training on ICSD crystals directly. Our results demonstrate that synthetically generated crystals can be
%used to extract structural information from ICSD powder diffractograms, which makes it possible to apply very large state-of-the-art machine learning models in the area of powder X-ray diffraction.
%We further show first steps toward applying our methodology to experimental data, where automated XRD data analysis is crucial, especially in high-throughput settings.
%While we focused on the prediction of the space group, our approach has the potential to be extended to related tasks in the future.