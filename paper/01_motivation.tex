
% Autonomous laboratory
% https://www.nature.com/articles/s41586-023-06734-w

% Accelerating material discovery
% https://www.nature.com/articles/s41524-022-00765-z

%High-throughput experiments facilitate materials innovation: A review
%https://link.springer.com/article/10.1007/s11431-018-9369-9#:~:text=High%2Dthroughput%20materials%20experiments%20including,at%20a%20fraction%20of%20cost.

The advent of high-throughput experiments, holds the prospect of accelerating the speed materials discovery multiple times over.
The synthesis and probing of novel materials is becoming increasingly efficient and automated,
which has correspondingly increased the throughput of samples in these steps of the experimentation pipeline.

% Rietveld Refinement (3.9 Profile fitting and rietveld analysis)
%https://www.sciencedirect.com/topics/biochemistry-genetics-and-molecular-biology/rietveld-refinement

% Rietveld Refinement: Half a century anniversary
% https://pubs.acs.org/doi/10.1021/acs.cgd.1c00854


% On Rietveld refinement
The objective of powder x-ray diffraction experiments is to extract as much information about the crystal structure
of the sample as possible, or if the sample is multi-phase, also identify the phases present in the sample and quantify
their fractions.
This is done through a technique called Rietveld refinement: A working model of the crystal structure, starting
from an initial guess provided by the operator of the analysis software, is assumed and the corresponding diffractogram
is simulated.
The parameters of the working model of the sample are then adjusted to minimize the difference between the simulated
diffraction pattern and the pattern observed in experiment.
Critically, this process does not generally guarantee to guide the working model towards a global minimum of the difference
in diffractograms and hence arrive at an accurate model of the sample.
In general one can only assume the refinement process to guide the working model towards the nearest local minimum
starting from the initial guess.
Hence, the result of the refinement procedure is in general only as good as the initial guess it was provided with.
Still this workflow often works well, since usually the operator of the experiment will be able
to provide an at least somewhat accurate initial guess of the crystal structure.\\

However, employing manually operated Rietveld Refinement is not without issues.
It is certainly not scalable to the degree that would be required to keep up with advances
in throughput and efficiency in other steps of the experimentation pipeline, even just due to the manual effort involved.
But beyond the lacking scalability of this workflow we also wish to highlight the following issue:
Providing the analysis software with an initial guess of the structure requires manually or algorithmically
consulting a reference library of crystal structures with corresponding simulated or sometimes real diffractograms.
From that reference library a crystal structure must be identified that is likely to be similar
to that of the investigated sample which can then serve as the initial guess provided to the analysis software.
Since the number of chemically viable crystal structures is infinite, any such library will always be incomplete.
The best that can be done to remedy this is to continually update the library both on the side
of the distributor and the user, which is an undesirable overhead.
Especially for novel material discovery, it may be impossible to find a sufficiently accurate match in the reference
library to apply Rietveld Refinement successfully.

Machine learning is a very promising candidate to eliminate both the need for manual inputs and the need for reference
libraries in powder x-ray diffraction analysis, resulting in a streamlined and automated analysis process that can
keep pace with an automated high-throughput experimentation environment. \\

Models can be trained to predict crystal structure properties given only the diffractogram itself, given that
the model can be trained on dataset consisting of pairs of diffractograms and corresponding crystal structures.
So far however, due to an absence of sufficiently large labeled datasets with diffractograms from experiment,
machine learning in this domain has largely relied on simulated diffractograms.
Note, that reference libraries which are used for Rietveld Refinement by in large use simulated diffractograms
rather than diffractograms from experiment and hence cannot either provide a dataset with diffractograms coming
from experiment. \\

Models trained on datasets with simulated diffractograms have already shown very strong performance in predicting
lattice parameters, spacegroup and crystallite size from simulated diffractograms but the performance drops off immensely
when these models are applied to data originating from experiment.
There is every reason to believe, that the strong performance observed in the domain of purely simulated diffractograms
can translate to predicting crystal structure properties of diffractograms originating from experiment, given that
models are trained on a dataset that accurately reflects diffractograms as they appear in experiment. \\

However due to the scarcity of labeled diffractograms originating from experiment, collecting a labeled dataset
of real experimental powder x-ray diffraction data of sufficient size to train a model
that can be useful in practice is next to impossible and is not the aim of this dataset.
Rather we hope that this dataset can enable machine learning researchers to evaluate how closely their simulations
represent actual experimental data, identify what features are missing and modify their simulations algorithms
accordingly to arrive at a simulation that nearly perfectly mirrors how data appears in reality.
Secondly we would like to also present this dataset as a challenge: The fraction of labeled data can be used to
easily test the performance of the model on a broad range of data as it appears in reality, which can be used
to gauge how well models would perform in practice. 



% Reference library required and analysis software required, reference library may be incomplete
% software training required, doesnt always converge or not to true minimum,
% sometimes sample is truely unknown,  manual labor required in any case

% -> This is too detailed for this paper, but can use it for thesis
%For one, under normal circumstances, the initial guess of the structure requires manually or computationally consulting
%a reference library of pairs of patterns and structures.
%The reference library is an undesirable overhead that needs to be distributed along with the analysis software and
%maintained to remain up to date and interoperable with the anlysis software.
%The industry standard in terms of such reference libraries is the Powder Diffraction File and costs around 1000\$ per
%year to license from its maintainer, the International Centre for Diffraction Data.
%Another issue is that since the number of chemically viable crystal structures is infinite any reference library, no
%matter how




% Crystallographic open database
% https://www.crystallography.net/cod/




Machine learning techniques have emerged as a powerful tool in the toolkit of
materials scientists. While they are often used to make predictions on
the properties of materials or find new materials with certain properties, an increasingly interesting domain is the automated analysis of raw experimental measurements
guided by machine learning\supercite{radovicMachineLearningEnergy2018}.

With the advent of high-throughput experiments, the amount of gathered data is
vast and the analysis often becomes a bottleneck in the processing pipeline
\supercite{rahmanianEnablingModularAutonomous2022}. Powder X-ray diffraction
(XRD) is an important measurement technique used to obtain structural
information from polycrystalline samples\supercite{harrisContemporaryAdvancesUse2001}. 
The diffractograms are an
information-dense fingerprint of the structure of the material. However,
analyzing these diffractograms is not an easy task\supercite{holderTutorialPowderXray2019}. 
Full structure solutions and
Rietveld refinement take time and require expert knowledge, both about the
analysis technique and the materials class at hand. This is not feasible in high-throughput experiments on a larger
scale. Therefore, the question arises whether it is possible to automatically 
analyze powder diffractograms with machine learning models trained on large
amounts of data, making it possible to run inference almost instantaneously.

During the last few years, there have been several studies tackling this
objective by applying machine learning models to various tasks concerning the
analysis of powder diffractograms such as phase
classification\supercite{leeDeeplearningTechniquePhase2020,maffettoneCrystallographyCompanionAgent2021,schuetzkeEnhancingDeeplearningTraining2021,szymanskiProbabilisticDeepLearning2021,wangRapidIdentificationXray2020},
phase fraction determination\supercite{leeDatadrivenXRDAnalysis2021}, space
group
classification\supercite{parkClassificationCrystalStructure2017,oviedoFastInterpretableClassification2019,zalogaCrystalSymmetryClassification2020,vecseiNeuralNetworkBased2019,suzukiSymmetryPredictionKnowledge2020,chakrabortyDeepCrystalStructure2022},
machine-learning-guided Rietveld
refinement\supercite{ozakiAutomatedCrystalStructure2020,fengMethodArtificialIntelligence2019},
extraction of lattice
parameters\supercite{dongDeepConvolutionalNeural2021,chitturiAutomatedPredictionLattice2021,chakrabortyDeepCrystalStructure2022,
habershonPowderDiffractionIndexing2004} and crystallite
sizes\supercite{dongDeepConvolutionalNeural2021,chakrabortyDeepCrystalStructure2022},
and also novelty detection based on unsupervised techniques\supercite{bankoDeepLearningVisualization2021}.
Since an abundant source of experimental diffractograms is hard to come by, most
applications train their models on simulated diffractograms from the Inorganic
Crystal Structure Database (ICSD)\supercite{bergerhoff1987}, which contains a total of 272\,260 structures (October 2022).

\citeauthor{leeDeeplearningTechniquePhase2020} used a deep convolutional neural
network (CNN) trained on a large dataset of multiphase compositions from the
quaternary \ce{Sr-Li-Al-O} pool to classify present phases in the
diffractogram\supercite{leeDeeplearningTechniquePhase2020}. In a follow-up
study, they further showed good results for phase fraction inference in the
quaternary \ce{Li-La-Zr-O} pool\supercite{leeDatadrivenXRDAnalysis2021}.
\citeauthor{schuetzkeEnhancingDeeplearningTraining2021} performed phase
classification on iron ores and cement compounds and used data augmentation with
respect to lattice parameters, crystallite sizes, and preferred orientation
\supercite{schuetzkeEnhancingDeeplearningTraining2021}. They showed that
especially the lattice parameter variations enhance the classification accuracy
significantly.

Instead of the analysis of phase composition,
\citeauthor{dongDeepConvolutionalNeural2021} performed regression of scale
factors, lattice parameters, and crystallite sizes in a five-phase catalytic
materials system\supercite{dongDeepConvolutionalNeural2021}. In contrast to
supervised tasks, \citeauthor{bankoDeepLearningVisualization2021} used a
variational autoencoder to visualize variations in space group, preferred
orientation, crystallite size, and peak
shifts\supercite{bankoDeepLearningVisualization2021}.
\citeauthor{parkClassificationCrystalStructure2017} used a deep CNN to classify
space groups of single-phase diffractograms, reaching a test accuracy of 81.14\%
on simulated diffractograms.\supercite{parkClassificationCrystalStructure2017}
However, as we will show later in this paper, this accuracy is highly overestimated and drops to 56.1\% when test splits are designed in a way to reduce data leakage in non-IID datasets such as the ICSD.
\citeauthor{vecseiNeuralNetworkBased2019}\supercite{vecseiNeuralNetworkBased2019} developed
a similar approach and applied their classifier to experimental diffractograms
from the RRUFF mineral database\supercite{lafuentePowerDatabasesRRUFF2015},
reaching an experimental test accuracy of 54\%.


% PDF pricing
%https://www.icdd.com/assets/files/2023-Purchase-Options.pdf

% Computationally intensive -> Probably outdated. If it used to take minutes in 2010
% it probably only takes seconds by now.
% https://www.rigaku.com/de/node/707

% A simple solution to the Rietveld refinement recipe problem
% -> Rietveld refinement requires "expert input"
% https://www.ncbi.nlm.nih.gov/pmc/articles/PMC10840306/

% Comments -> Why having to do rietveld refinement sucks
% FullProf tutorial on fitting XRD peaks -> Tedious, requires lots of manual input
% https://pranabdas.github.io/research/fullprof/

% Topas tutorial -> 44 page tutorial, many boring and potentially frustrating steps, lots of points offer potential for task failure
% https://topas.webspace.durham.ac.uk/topas_user_menu/

%  Methods and Tutorials – Powder Diffraction -> Overview of rietveld refinement data
%https://neutrons.ornl.gov/sites/default/files/Methods%20and%20Tutorials%20Powder%20Diffraction.pdf

% Rietveld Refinement from Powder Diffraction Data -> 50 pg tutorial that no one wants to read, overwhelming amout of quantities/figures that needs to be manually looked at in the process
% https://www.fkf.mpg.de/4112052/cpd26.pdf
